\documentclass[compress, aspectratio=54]{beamer}
%\documentclass[notes=show]{beamer}
%\documentclass[xcolor=dvipsnames]{beamer}
\usepackage[export]{adjustbox}
\usepackage{sidecap}
\usepackage{subfig}
\usepackage{amssymb}
\usepackage{latexsym}
\usepackage{amsfonts}
\usepackage{amsmath}
\usepackage[absolute,overlay]{textpos}
\usepackage[english]{babel}
\usepackage[latin1]{inputenc}
\usepackage{subfig}
%\usepackage{times}
\usepackage[T1]{fontenc}
\usepackage{tabularx}
\newcolumntype{Y}{>{\small\raggedright\arraybackslash}X}
\usepackage{graphicx}
\usepackage{bigstrut}
\usepackage{bbm}
\usepackage{mathrsfs}
\usepackage{epsfig}
\usepackage{array}
%\usepackage{natbib}
\usepackage{hyperref}
\usepackage{caption}
\usepackage{comment}

\mode<presentation> {
%\usetheme[left,width=1.7cm]{Berkeley}
%\usetheme{default}
\usetheme{Boadilla}
  \usecolortheme[RGB={103,102,204}]{structure}
%\usecolortheme{dove}
  \useoutertheme{infolines}
  \setbeamercovered{transparent}
 }

%\usepackage[utf8]{inputenc}

% Default fixed font does not support bold face
\DeclareFixedFont{\ttb}{T1}{txtt}{bx}{n}{12} % for bold
\DeclareFixedFont{\ttm}{T1}{txtt}{m}{n}{12}  % for normal

% Custom colors
\usepackage{color}
\definecolor{deepblue}{rgb}{0,0,0.5}
\definecolor{deepred}{rgb}{0.6,0,0}
\definecolor{deepgreen}{rgb}{0,0.5,0}

\usepackage{listings}

% Python style for highlighting
\newcommand\pythonstyle{\lstset{
language=Python,
basicstyle=\ttm,
otherkeywords={self},             % Add keywords here
keywordstyle=\ttb\color{deepblue},
emph={MyClass,__init__},          % Custom highlighting
emphstyle=\ttb\color{deepred},    % Custom highlighting style
stringstyle=\color{deepgreen},
frame=tb,                         % Any extra options here
showstringspaces=false            % 
}}


% Python environment
\lstnewenvironment{python}[1][]
{
\pythonstyle
\lstset{#1}
}
{}

% Python for external files
\newcommand\pythonexternal[2][]{{
\pythonstyle
\lstinputlisting[#1]{#2}}}

% Python for inline
\newcommand\pythoninline[1]{{\pythonstyle\lstinline!#1!}}
%\renewcommand{\familydefault}{cmss}
%\renewcommand{\mathrm}{\mathsf}
%\renewcommand{\textrm}{\textsf}
\usefonttheme{serif}
\newcommand{\X}{{\mathbf{X}}}
\newcommand{\x}{{\mathbf{x}}}
\newcommand{\E}{\mathsf{E}}
\newcommand{\V}{\mathsf{Var}}

\DeclareGraphicsExtensions{.jpg,.pdf,.mps,.png}

\setbeamercolor{bibliography entry title}{fg=black}
\setbeamercolor{bibliography entry author}{fg=black}
\setbeamercolor{subsection in toc}{fg=structure}
\setbeamercolor{palette primary}{bg=structure, fg=white}
%\setbeamercolor{palette secondary}{bg=structure, fg=black}
%\setbeamercolor{palette tertiary}{bg=structure, fg=black}
\setbeamercolor{caption name}{fg=black} \setbeamersize{text margin
left=.8cm} \setbeamersize{text margin right=1cm}
\hypersetup{linkbordercolor={1 0 0}} \setbeamertemplate{navigation
symbols}{} \setbeamertemplate{headline}[default]

\setbeamertemplate{enumerate items}[default]

\newcounter{transfct}
\newcounter{begbs}
\newcounter{endbs}


\title[Introduction]{Introduction to Python}

\author[Arieda Mu\c co]{Arieda Mu\c co}
\institute[CEU]{Central European University}

\AtBeginSection[] {
  \begin{frame}<handout:0>
    \frametitle{TOC}
    \tableofcontents[currentsection]
  \end{frame}
}

\date{}

\pgfdeclareimage[height=.7cm]{logo}{rgs2}
\logo{\pgfuseimage{logo}}
\begin{document}
\captionsetup[subfigure]{labelformat=empty}

\frame{\titlepage}

%%%%%%%%%%%%%%%%%%%%%%%%%%%%%%%%%%%%%%%%%%%


\begin{frame}
\frametitle{Information}
\begin{itemize}
\item My research focuses on two areas: Political and Development Economics. In my research, I deal with tons of data and (lots of) text data -> programming with Python. That's why this course.
\item Introduce yourself. What are your expectations? Why are you here? What kind of data you are currently using or plan to use? 
\end{itemize}
\end{frame}
%----------------------------------------------------------------------------%

\begin{frame}
\frametitle{Plan for this course}
\begin{itemize}
\item Introduction to Python foundations
\end{itemize}
\end{frame}
%----------------------------------------------------------------------------%

\begin{frame}
\frametitle{The team }
\begin{itemize}
\item Arieda Mu\c co: \href{mailto:MucoA@ceu.edu}{MucoA@ceu.edu}. Office: Quellenstrasse,  51\\
\item Adam Nasli (TA): \href{mailto:adam.nasli@brokerchooser.com}{adam.nasli@brokerchooser.com}\\
\begin{figure}%
   
    \subfloat[Arieda]{{\includegraphics[width=2.7cm,height=3.5cm]{../Figures/mucoa} }}%
        \qquad
    \subfloat[Adam]{{\includegraphics[width=2.7cm,height=3.5cm]{../Figures/adam.png} }}%
\end{figure}
\end{itemize}
We encourage you to ask questions via Slack. When needed we'll set meetings via Zoom.

\end{frame}


%----------------------------------------------------------------------------%


\begin{frame}
\frametitle{Grading}
Final assessment will consist of the following:
\begin{itemize}
\item \textbf{Quizzes in Class} (20\% of final grade)
\item \textbf{Problem Sets} (40\% of final grade)
\item \textbf{Individual Project}  (40\% of final grade)

I expect most cameras on. If less than 60 percent of the class has their cameras off in a class, all participants will be deducted 1 point. Experiment with virtual backgrounds.
\end{itemize}
\end{frame}

%----------------------------------------------------------------------------%

%----------------------------------------------------------------------------%


\begin{frame}
\frametitle{Deadlines}

\begin{itemize}
\item Past deadline submissions do not get graded
\item Email for meetings, questions etc 
\item Emails/Questions: You will get a reply if you send an email but send it 24 hours before a deadline (no response otherwise)
\item Slack will be our communication tool for this course
\begin{itemize}
\item Post questions and answers in respective channels
\item Keep a close eye on channels on quizzes and assignments
\item Make sure you reply in thread when needed.
\end{itemize}
\item We strongly encourage peer learning. Feel free to post in the Slack channel if you think some information is of common interest
\end{itemize}

\end{frame}

%----------------------------------------------------------------------------%
\begin{frame}
\frametitle{Rules}
\begin{itemize}
\item Ask questions and feel free to google
\begin{itemize}
\item Don't feel bad about this. Even software developers spend a lot of their coding time googling programming related questions
\item Important to know how to read error messages
\begin{itemize}

\item or google them
\end{itemize}
\item Stack Overflow is a programmer's best friend
\end{itemize}
\end{itemize}
\end{frame}

%----------------------------------------------------------------------------%

\begin{frame}
\begin{figure}

\includegraphics[width=0.6\linewidth ]{../Figures/stack-overflow.jpeg}
\end{figure}

\end{frame}

%----------------------------------------------------------------------------%
\begin{frame}
\frametitle{Recommended Material}
\begin{itemize}
\item \href{https://www.codecademy.com/catalog/language/python}{\color{ blue}{Codecademy}} is the place to start
\item \href{https://automatetheboringstuff.com/}{\color{ blue}{Automate the Boring Stuff with Python}} and \href{https://realpython.com/}{\color{ blue}{The Real Python}} are great sources
\item A Shaw, Zed. "Learn Python the hard way"
\item Al Sweigart. "Automate the Boring Stuff with Python"
\item Allen B. Downey. "Think Python: How to Think Like a Computer Scientist"
\end{itemize}
\end{frame}



\begin{frame}
\frametitle{A bit about Python}
\begin{itemize}

\item Programming language intended for general-purpose high-level language
\item Web development, scientific and numeric education, desktop graphical user interface, software development
\item Free and open source 
\item You can do everything that you can do in a programming language
\item Big community (Google, Youtube, Nasa...)
\item High readability (more than R or C)
\item Python was first released in early 1980
\begin{itemize}

\item Python 2 in 2000 and Python 3 in 2008
\end{itemize}
\end{itemize}

\end{frame}
%----------------------------------------------------------------------------%
%----------------------------------------------------------------------------%

\begin{frame}
\frametitle{Black Holes and Python}
\begin{figure}

\includegraphics[width=0.5\linewidth ]{../Figures/black_hole.png}
\end{figure}

\end{frame}

%----------------------------------------------------------------------------%
\begin{frame}
\frametitle{Annoying things in Python}
\begin{itemize}

\item Python 3 is not backward compatible with Python 2
\begin{itemize}
\item In this course we will use Python 3. Python 2 is not supported anymore
\item If you are starting a new project, do so in Python 3
\end{itemize}

\item Pandas Library (more on this next time)
\begin{itemize}
\item But very useful 
\end{itemize}
\item + some minor things we'll cover throughout the course 
\begin{itemize}

\item example: split() vs join()
\begin{itemize}
\item sentence = "We will rock you!"

\item words = sentence.split(" ") but sentence = " ".join(words) (?)
\end{itemize}


\end{itemize}

\end{itemize}

\end{frame}
%----------------------------------------------------------------------------%
\begin{frame}
\frametitle{Purpose of the course}
\begin{itemize}

\item Programming in Python is (mildly put) very broad topics, and we will not be able to cover many(!) things
\item Build strong foundations such that in the future you get confidence in starting to dig deeper into these topics
 
\end{itemize}

\end{frame}
%----------------------------------------------------------------------------%

%----------------------------------------------------------------------------%
\end{document}

